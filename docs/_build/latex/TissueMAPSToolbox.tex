% Generated by Sphinx.
\def\sphinxdocclass{report}
\documentclass[letterpaper,10pt,english]{sphinxmanual}
\usepackage[utf8]{inputenc}
\DeclareUnicodeCharacter{00A0}{\nobreakspace}
\usepackage{cmap}
\usepackage[T1]{fontenc}
\usepackage{babel}
\usepackage{times}
\usepackage[Bjarne]{fncychap}
\usepackage{longtable}
\usepackage{sphinx}
\usepackage{multirow}

\addto\captionsenglish{\renewcommand{\figurename}{Fig. }}
\addto\captionsenglish{\renewcommand{\tablename}{Table }}
\floatname{literal-block}{Listing }



\title{TissueMAPS Toolbox Documentation}
\date{June 27, 2015}
\release{0.1.0}
\author{Markus D. Herrmann, Robin Hafen}
\newcommand{\sphinxlogo}{\includegraphics{logo.png}\par}
\renewcommand{\releasename}{Release}
\makeindex

\makeatletter
\def\PYG@reset{\let\PYG@it=\relax \let\PYG@bf=\relax%
    \let\PYG@ul=\relax \let\PYG@tc=\relax%
    \let\PYG@bc=\relax \let\PYG@ff=\relax}
\def\PYG@tok#1{\csname PYG@tok@#1\endcsname}
\def\PYG@toks#1+{\ifx\relax#1\empty\else%
    \PYG@tok{#1}\expandafter\PYG@toks\fi}
\def\PYG@do#1{\PYG@bc{\PYG@tc{\PYG@ul{%
    \PYG@it{\PYG@bf{\PYG@ff{#1}}}}}}}
\def\PYG#1#2{\PYG@reset\PYG@toks#1+\relax+\PYG@do{#2}}

\expandafter\def\csname PYG@tok@gd\endcsname{\def\PYG@tc##1{\textcolor[rgb]{0.63,0.00,0.00}{##1}}}
\expandafter\def\csname PYG@tok@gu\endcsname{\let\PYG@bf=\textbf\def\PYG@tc##1{\textcolor[rgb]{0.50,0.00,0.50}{##1}}}
\expandafter\def\csname PYG@tok@gt\endcsname{\def\PYG@tc##1{\textcolor[rgb]{0.00,0.27,0.87}{##1}}}
\expandafter\def\csname PYG@tok@gs\endcsname{\let\PYG@bf=\textbf}
\expandafter\def\csname PYG@tok@gr\endcsname{\def\PYG@tc##1{\textcolor[rgb]{1.00,0.00,0.00}{##1}}}
\expandafter\def\csname PYG@tok@cm\endcsname{\let\PYG@it=\textit\def\PYG@tc##1{\textcolor[rgb]{0.25,0.50,0.56}{##1}}}
\expandafter\def\csname PYG@tok@vg\endcsname{\def\PYG@tc##1{\textcolor[rgb]{0.73,0.38,0.84}{##1}}}
\expandafter\def\csname PYG@tok@m\endcsname{\def\PYG@tc##1{\textcolor[rgb]{0.13,0.50,0.31}{##1}}}
\expandafter\def\csname PYG@tok@mh\endcsname{\def\PYG@tc##1{\textcolor[rgb]{0.13,0.50,0.31}{##1}}}
\expandafter\def\csname PYG@tok@cs\endcsname{\def\PYG@tc##1{\textcolor[rgb]{0.25,0.50,0.56}{##1}}\def\PYG@bc##1{\setlength{\fboxsep}{0pt}\colorbox[rgb]{1.00,0.94,0.94}{\strut ##1}}}
\expandafter\def\csname PYG@tok@ge\endcsname{\let\PYG@it=\textit}
\expandafter\def\csname PYG@tok@vc\endcsname{\def\PYG@tc##1{\textcolor[rgb]{0.73,0.38,0.84}{##1}}}
\expandafter\def\csname PYG@tok@il\endcsname{\def\PYG@tc##1{\textcolor[rgb]{0.13,0.50,0.31}{##1}}}
\expandafter\def\csname PYG@tok@go\endcsname{\def\PYG@tc##1{\textcolor[rgb]{0.20,0.20,0.20}{##1}}}
\expandafter\def\csname PYG@tok@cp\endcsname{\def\PYG@tc##1{\textcolor[rgb]{0.00,0.44,0.13}{##1}}}
\expandafter\def\csname PYG@tok@gi\endcsname{\def\PYG@tc##1{\textcolor[rgb]{0.00,0.63,0.00}{##1}}}
\expandafter\def\csname PYG@tok@gh\endcsname{\let\PYG@bf=\textbf\def\PYG@tc##1{\textcolor[rgb]{0.00,0.00,0.50}{##1}}}
\expandafter\def\csname PYG@tok@ni\endcsname{\let\PYG@bf=\textbf\def\PYG@tc##1{\textcolor[rgb]{0.84,0.33,0.22}{##1}}}
\expandafter\def\csname PYG@tok@nl\endcsname{\let\PYG@bf=\textbf\def\PYG@tc##1{\textcolor[rgb]{0.00,0.13,0.44}{##1}}}
\expandafter\def\csname PYG@tok@nn\endcsname{\let\PYG@bf=\textbf\def\PYG@tc##1{\textcolor[rgb]{0.05,0.52,0.71}{##1}}}
\expandafter\def\csname PYG@tok@no\endcsname{\def\PYG@tc##1{\textcolor[rgb]{0.38,0.68,0.84}{##1}}}
\expandafter\def\csname PYG@tok@na\endcsname{\def\PYG@tc##1{\textcolor[rgb]{0.25,0.44,0.63}{##1}}}
\expandafter\def\csname PYG@tok@nb\endcsname{\def\PYG@tc##1{\textcolor[rgb]{0.00,0.44,0.13}{##1}}}
\expandafter\def\csname PYG@tok@nc\endcsname{\let\PYG@bf=\textbf\def\PYG@tc##1{\textcolor[rgb]{0.05,0.52,0.71}{##1}}}
\expandafter\def\csname PYG@tok@nd\endcsname{\let\PYG@bf=\textbf\def\PYG@tc##1{\textcolor[rgb]{0.33,0.33,0.33}{##1}}}
\expandafter\def\csname PYG@tok@ne\endcsname{\def\PYG@tc##1{\textcolor[rgb]{0.00,0.44,0.13}{##1}}}
\expandafter\def\csname PYG@tok@nf\endcsname{\def\PYG@tc##1{\textcolor[rgb]{0.02,0.16,0.49}{##1}}}
\expandafter\def\csname PYG@tok@si\endcsname{\let\PYG@it=\textit\def\PYG@tc##1{\textcolor[rgb]{0.44,0.63,0.82}{##1}}}
\expandafter\def\csname PYG@tok@s2\endcsname{\def\PYG@tc##1{\textcolor[rgb]{0.25,0.44,0.63}{##1}}}
\expandafter\def\csname PYG@tok@vi\endcsname{\def\PYG@tc##1{\textcolor[rgb]{0.73,0.38,0.84}{##1}}}
\expandafter\def\csname PYG@tok@nt\endcsname{\let\PYG@bf=\textbf\def\PYG@tc##1{\textcolor[rgb]{0.02,0.16,0.45}{##1}}}
\expandafter\def\csname PYG@tok@nv\endcsname{\def\PYG@tc##1{\textcolor[rgb]{0.73,0.38,0.84}{##1}}}
\expandafter\def\csname PYG@tok@s1\endcsname{\def\PYG@tc##1{\textcolor[rgb]{0.25,0.44,0.63}{##1}}}
\expandafter\def\csname PYG@tok@gp\endcsname{\let\PYG@bf=\textbf\def\PYG@tc##1{\textcolor[rgb]{0.78,0.36,0.04}{##1}}}
\expandafter\def\csname PYG@tok@sh\endcsname{\def\PYG@tc##1{\textcolor[rgb]{0.25,0.44,0.63}{##1}}}
\expandafter\def\csname PYG@tok@ow\endcsname{\let\PYG@bf=\textbf\def\PYG@tc##1{\textcolor[rgb]{0.00,0.44,0.13}{##1}}}
\expandafter\def\csname PYG@tok@sx\endcsname{\def\PYG@tc##1{\textcolor[rgb]{0.78,0.36,0.04}{##1}}}
\expandafter\def\csname PYG@tok@bp\endcsname{\def\PYG@tc##1{\textcolor[rgb]{0.00,0.44,0.13}{##1}}}
\expandafter\def\csname PYG@tok@c1\endcsname{\let\PYG@it=\textit\def\PYG@tc##1{\textcolor[rgb]{0.25,0.50,0.56}{##1}}}
\expandafter\def\csname PYG@tok@kc\endcsname{\let\PYG@bf=\textbf\def\PYG@tc##1{\textcolor[rgb]{0.00,0.44,0.13}{##1}}}
\expandafter\def\csname PYG@tok@c\endcsname{\let\PYG@it=\textit\def\PYG@tc##1{\textcolor[rgb]{0.25,0.50,0.56}{##1}}}
\expandafter\def\csname PYG@tok@mf\endcsname{\def\PYG@tc##1{\textcolor[rgb]{0.13,0.50,0.31}{##1}}}
\expandafter\def\csname PYG@tok@err\endcsname{\def\PYG@bc##1{\setlength{\fboxsep}{0pt}\fcolorbox[rgb]{1.00,0.00,0.00}{1,1,1}{\strut ##1}}}
\expandafter\def\csname PYG@tok@mb\endcsname{\def\PYG@tc##1{\textcolor[rgb]{0.13,0.50,0.31}{##1}}}
\expandafter\def\csname PYG@tok@ss\endcsname{\def\PYG@tc##1{\textcolor[rgb]{0.32,0.47,0.09}{##1}}}
\expandafter\def\csname PYG@tok@sr\endcsname{\def\PYG@tc##1{\textcolor[rgb]{0.14,0.33,0.53}{##1}}}
\expandafter\def\csname PYG@tok@mo\endcsname{\def\PYG@tc##1{\textcolor[rgb]{0.13,0.50,0.31}{##1}}}
\expandafter\def\csname PYG@tok@kd\endcsname{\let\PYG@bf=\textbf\def\PYG@tc##1{\textcolor[rgb]{0.00,0.44,0.13}{##1}}}
\expandafter\def\csname PYG@tok@mi\endcsname{\def\PYG@tc##1{\textcolor[rgb]{0.13,0.50,0.31}{##1}}}
\expandafter\def\csname PYG@tok@kn\endcsname{\let\PYG@bf=\textbf\def\PYG@tc##1{\textcolor[rgb]{0.00,0.44,0.13}{##1}}}
\expandafter\def\csname PYG@tok@o\endcsname{\def\PYG@tc##1{\textcolor[rgb]{0.40,0.40,0.40}{##1}}}
\expandafter\def\csname PYG@tok@kr\endcsname{\let\PYG@bf=\textbf\def\PYG@tc##1{\textcolor[rgb]{0.00,0.44,0.13}{##1}}}
\expandafter\def\csname PYG@tok@s\endcsname{\def\PYG@tc##1{\textcolor[rgb]{0.25,0.44,0.63}{##1}}}
\expandafter\def\csname PYG@tok@kp\endcsname{\def\PYG@tc##1{\textcolor[rgb]{0.00,0.44,0.13}{##1}}}
\expandafter\def\csname PYG@tok@w\endcsname{\def\PYG@tc##1{\textcolor[rgb]{0.73,0.73,0.73}{##1}}}
\expandafter\def\csname PYG@tok@kt\endcsname{\def\PYG@tc##1{\textcolor[rgb]{0.56,0.13,0.00}{##1}}}
\expandafter\def\csname PYG@tok@sc\endcsname{\def\PYG@tc##1{\textcolor[rgb]{0.25,0.44,0.63}{##1}}}
\expandafter\def\csname PYG@tok@sb\endcsname{\def\PYG@tc##1{\textcolor[rgb]{0.25,0.44,0.63}{##1}}}
\expandafter\def\csname PYG@tok@k\endcsname{\let\PYG@bf=\textbf\def\PYG@tc##1{\textcolor[rgb]{0.00,0.44,0.13}{##1}}}
\expandafter\def\csname PYG@tok@se\endcsname{\let\PYG@bf=\textbf\def\PYG@tc##1{\textcolor[rgb]{0.25,0.44,0.63}{##1}}}
\expandafter\def\csname PYG@tok@sd\endcsname{\let\PYG@it=\textit\def\PYG@tc##1{\textcolor[rgb]{0.25,0.44,0.63}{##1}}}

\def\PYGZbs{\char`\\}
\def\PYGZus{\char`\_}
\def\PYGZob{\char`\{}
\def\PYGZcb{\char`\}}
\def\PYGZca{\char`\^}
\def\PYGZam{\char`\&}
\def\PYGZlt{\char`\<}
\def\PYGZgt{\char`\>}
\def\PYGZsh{\char`\#}
\def\PYGZpc{\char`\%}
\def\PYGZdl{\char`\$}
\def\PYGZhy{\char`\-}
\def\PYGZsq{\char`\'}
\def\PYGZdq{\char`\"}
\def\PYGZti{\char`\~}
% for compatibility with earlier versions
\def\PYGZat{@}
\def\PYGZlb{[}
\def\PYGZrb{]}
\makeatother

\renewcommand\PYGZsq{\textquotesingle}

\begin{document}

\maketitle
\tableofcontents
\phantomsection\label{index::doc}


Contents:


\chapter{Introduction}
\label{intro:introduction}\label{intro:welcome-to-tissuemaps-toolbox-s-documentation}\label{intro::doc}\label{intro:id1}
\textbf{tmt} is a Python package that bundles image processing and data analysis tools for \href{https://github.com/HackerMD/TissueMAPS}{TissueMAPS}.


\section{Subpackages}
\label{intro:subpackages}\label{intro:id2}
\textbf{align} - Alignment of images between different acquisition cycles.

\textbf{corilla} - Correction of illumination artifacts.

\textbf{dafu} - Data fusion for \href{https://github.com/HackerMD/Jterator}{Jterator} projects.

\textbf{illuminati} - Creation of image pyramids.

\textbf{visi} - Conversion of Visitron's STK files to PNG images with optional renaming.


\section{Configuration settings}
\label{intro:configurationsettings}\label{intro:configuration-settings}
Configurations are defined in \emph{.config} \href{http://yaml.org/}{YAML} files to specify the experiment layout, such as the directory structure on disk.

Paths and filenames are described with \href{https://docs.python.org/2/library/string.html\#formatstrings}{Python format strings}. The \textbf{replacement fields} surrounded by curly braces \code{\{\}} are then automatically replaced with experiment specific variables.
\begin{description}
\item[{To this end, you can use the following \emph{replacement fields}:}] \leavevmode\begin{itemize}
\item {} 
\emph{experiment\_dir}: absolute path to the experiment directory

\item {} 
\emph{experiment}: name of the experiment folder

\item {} 
\emph{subexperiment}: name of a subexperiment folder, i.e. a subfolder of the experiment folder

\item {} 
\emph{cycle}: number of a subexperiment

\item {} 
\emph{channel}: number of a channel of intensity images (layers)

\item {} 
\emph{objects}: name of objects in segmentation images (masks)

\end{itemize}

\end{description}

\begin{Verbatim}[commandchars=\\\{\}]
SUBEXPERIMENTS\PYGZus{}EXIST: Yes

\PYGZsh{} Path format strings
IMAGE\PYGZus{}FOLDER\PYGZus{}LOCATION: \PYGZsq{}\PYGZob{}experiment\PYGZus{}dir\PYGZcb{}/\PYGZob{}subexperiment\PYGZcb{}/images\PYGZsq{}
SHIFT\PYGZus{}FOLDER\PYGZus{}LOCATION: \PYGZsq{}\PYGZob{}experiment\PYGZus{}dir\PYGZcb{}/\PYGZob{}subexperiment\PYGZcb{}/shift\PYGZsq{}
STATS\PYGZus{}FOLDER\PYGZus{}LOCATION: \PYGZsq{}\PYGZob{}experiment\PYGZus{}dir\PYGZcb{}/\PYGZob{}subexperiment\PYGZcb{}/stats\PYGZsq{}
SEGMENTATION\PYGZus{}FOLDER\PYGZus{}LOCATION: \PYGZsq{}\PYGZob{}experiment\PYGZus{}dir\PYGZcb{}/\PYGZob{}subexperiment\PYGZcb{}/segmentations\PYGZsq{}
LAYERS\PYGZus{}FOLDER\PYGZus{}LOCATION: \PYGZsq{}\PYGZob{}experiment\PYGZus{}dir\PYGZcb{}/layers\PYGZsq{}
DATA\PYGZus{}FILE\PYGZus{}LOCATION: \PYGZsq{}\PYGZob{}experiment\PYGZus{}dir\PYGZcb{}/data.h5\PYGZsq{}

\PYGZsh{} Filename format strings
SUBEXPERIMENT\PYGZus{}FOLDER\PYGZus{}FORMAT: \PYGZsq{}\PYGZob{}experiment\PYGZcb{}\PYGZus{}\PYGZob{}cycle:0\PYGZgt{}2\PYGZcb{}\PYGZsq{}
SUBEXPERIMENT\PYGZus{}FILE\PYGZus{}FORMAT: \PYGZsq{}\PYGZob{}experiment\PYGZcb{}\PYGZus{}\PYGZob{}cycle\PYGZcb{}\PYGZsq{}
STATS\PYGZus{}FILE\PYGZus{}FORMAT: \PYGZsq{}illumstats\PYGZus{}channel\PYGZob{}channel:0\PYGZgt{}3\PYGZcb{}.h5\PYGZsq{}
SHIFT\PYGZus{}FILE\PYGZus{}FORMAT: \PYGZsq{}shiftDescriptor.json\PYGZsq{}

\PYGZsh{} Regular expression patterns to extract information encoded in filenames
EXPERIMENT\PYGZus{}FROM\PYGZus{}FILENAME: \PYGZsq{}\PYGZca{}([\PYGZca{}\PYGZus{}]+)\PYGZsq{}
CYCLE\PYGZus{}FROM\PYGZus{}FILENAME: \PYGZsq{}\PYGZus{}(\PYGZbs{}d+)\PYGZus{}\PYGZsq{}
COORDINATES\PYGZus{}FROM\PYGZus{}FILENAME: \PYGZsq{}\PYGZus{}r(\PYGZbs{}d+)\PYGZus{}c(\PYGZbs{}d+)\PYGZus{}\PYGZsq{}
COORDINATES\PYGZus{}IN\PYGZus{}FILENAME\PYGZus{}ONE\PYGZus{}BASED: Yes
SITE\PYGZus{}FROM\PYGZus{}FILENAME: \PYGZsq{}\PYGZus{}s(\PYGZbs{}d+)\PYGZus{}\PYGZsq{}
CHANNEL\PYGZus{}FROM\PYGZus{}FILENAME: \PYGZsq{}C(\PYGZbs{}d+)\PYGZbs{}.png\PYGZdl{}\PYGZsq{}
OBJECTS\PYGZus{}FROM\PYGZus{}FILENAME: \PYGZsq{}\PYGZus{}segmented(\PYGZbs{}w+).png\PYGZdl{}\PYGZsq{}

\PYGZsh{} Should Vips image processing library be used? Required for pyramid creation!
USE\PYGZus{}VIPS\PYGZus{}LIBRARY: Yes

\PYGZsh{} These settings are hard\PYGZhy{}coded in TissueMAPS, so don\PYGZsq{}t change them!
LAYERS\PYGZus{}FOLDER\PYGZus{}LOCATION: \PYGZsq{}\PYGZob{}experiment\PYGZus{}dir\PYGZcb{}/layers\PYGZsq{}
ID\PYGZus{}TABLES\PYGZus{}FOLDER\PYGZus{}LOCATION: \PYGZsq{}\PYGZob{}experiment\PYGZus{}dir\PYGZcb{}/id\PYGZus{}tables\PYGZsq{}
ID\PYGZus{}PYRAMIDS\PYGZus{}FOLDER\PYGZus{}LOCATION: \PYGZsq{}\PYGZob{}experiment\PYGZus{}dir\PYGZcb{}/id\PYGZus{}pyramids\PYGZsq{}
DATA\PYGZus{}FILE\PYGZus{}LOCATION: \PYGZsq{}\PYGZob{}experiment\PYGZus{}dir\PYGZcb{}/data.h5\PYGZsq{}
\end{Verbatim}
\begin{quote}

NOTE: Quotes are generally not required around strings in YAML syntax, but are necessary here because of the curly braces in the format strings!
\end{quote}


\section{Documentation}
\label{intro:documentation}\label{intro:id3}
\href{http://sphinx-doc.org/index.html}{Sphinx} is used for the documentation of source code in combination with the \href{https://pypi.python.org/pypi/sphinxcontrib-napoleon}{Napoleon extension} to support the \href{https://github.com/numpy/numpy/blob/master/doc/HOWTO\_DOCUMENT.rst.txt\#docstring-standard}{reStructuredText NumPy style}.

Documentation is located under \emph{docs} and will ultimately be hosted on \href{https://readthedocs.org/}{Read the Docs}.

To update the documentation upon changes in the source code, do

\begin{Verbatim}[commandchars=\\\{\}]
sphinx\PYGZhy{}apidoc \PYGZhy{}o ./docs ./tmt
\end{Verbatim}

To build HTML, do

\begin{Verbatim}[commandchars=\\\{\}]
cd docs
make html
\end{Verbatim}

The generated HTML files are located at \emph{docs/\_build/html}.


\chapter{Indices and tables}
\label{index:indices-and-tables}\begin{itemize}
\item {} 
\DUspan{xref,std,std-ref}{genindex}

\item {} 
\DUspan{xref,std,std-ref}{modindex}

\item {} 
\DUspan{xref,std,std-ref}{search}

\end{itemize}



\renewcommand{\indexname}{Index}
\printindex
\end{document}
